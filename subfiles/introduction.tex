\documentclass[../projects/thesis.tex]{subfiles}

\begin{document}

\chapter{Pendahuluan}\label{chap:pendahuluan}

\section{Latar Belakang}

%THE IMPORTANCE RIVERSIDE IN GENERAL
Menurut \cite{shamsuddin2013}, riverside merupakan ruang perkotaan yang harus terus berkembang. Kawasan inilah yang diberkahi dengan karakteristik dan perhatian khusus mengingat pentingnya air sebagai sumber kehidupan kota.
Pada area laut perkotaan, lomba untuk ruang waterfront, kebutuhan publik untuk mengakses pesisir laut dan mempertahankan biodiversity tepi laut sebagai sumber alami menjadi isu terhangat dalam kebijakan perkotaan \citep{breen1994waterfronts}.
%KEGAGALAN WATERFRONT IN GENERAL
Tidak hanya memberikan sebuah ruang pada suatu lahan di perkotaan, tetapi interaksi antara ruang dan pengguna adalah sangat penting. Pengembangan tepi laut seharusnya yang menciptakan ruang publik yang menyenangkan. Banyak kota yang telah gagal dalam memberikan kesan pada sebuah Waterfront, pandangan orang menjadi sosok yang kotor, padat, bahkan bahaya di sebuah sudut kota. Dalam pandangan \cite{goodwin1999} tepi laut biasa dipersepsikan sebagai sesuatu yang samar dan tercampur dengan area yang diabaikan, pusat komersial dan permukiman.

%PENURUNAN KUALITAS
Seiring perkembangan kota, waterfront seringkali mengalami kegoyahan dalam perkembangannya. Beberapa hal yang menyebabkan penurunan kualitas  waterfront adalah adanya peningkatan mobilitas yang melewati area tersebut\citep{richarda.lehmann1966}. \cite{ulam2009} menjelaskan kawasan waterfront pada paruh kedua abad ke-20 Waterfront harlem mengalami penurunan kualitas ditandai tempat hanya menjadi tempat memancing, penggunaan narkoba dan prostitusi. Bahkan menerima persepsi yang buruk oleh penduduk, tidak dapat di akses, \textit{privatisasi} bank, kontaminasi air dan pembukaan jalur tol paralel dengan pesisir \citep{shamsuddin2013}. Menurut \cite{benages2015revisiting} tepi laut mengalami seperti penurunan yang signifikan disebabkan oleh berbagai faktor seperti tekanan \textit{real estate} dan perencanaan yang kurang. Kejadian ini membuat pemerintah kesusahan dalam mengatur pengembangan tepi laut \citep{gripaios1999ports}.
%REDEVELOPMENT
Meski pun demikian, pengembangan waterfront kian meningkat dari abad 19. Misalnya tepi laut kota Witconsin yang mengalami kegagalan dalam memberikan rasa ruang dan perbedaan kota itu akhirnya berubah. Dengan konsep `Waterfront Renewal' melibatkan 50 kota di Amerika \citep{richarda.lehmann1966}. Ditambah lagi dengan `waterfront redevelopment' pada pelabuhan Thessaloniki di Yunani. Serta di Liverpool, inggris sekitar tahun 1940 \citep{couch2003city}. Adanya peningkatan ini memberikan harapan dan potensi untuk pengembangan tepi laut di tempat lainnya.
%WATERFRONT RESTORE CITY
Pesisir laut saat ini menjadi fokus pengembangan dalam sebuah kota. Dalam penelitian pelabuhan dan tepi laut industri di Thessaloniki, Yunani. Meski pun dalam keadaan ekonomi terpuruk, mereka masih menyediakan area yang cukup luas untuk bangunan mewah baru yang berdiri di pinggir laut. Pembangunan tersebut kebanyakan berada di pusat kota dan menjadi simbol suatu ekonomi 'sukses' \citep{vayona2011}.

% Redevelopment pinggir laut parepare
Pada tahun 2011, kota Parepare memulai perencanaan penataan pasar senggol serta jalan kawasan pinggir laut. Hal ini untuk merespon tujuan kota sebagai kota wisata yang berujung dalam peningkatan ekonomi dan nilai sebuah kota. Revitalisasi tersebut memunculkan beragam desain yang berhubungan dengan  preferensi pengunjung terhadap kawasan. Secara garis besar, pembagian macam kawasan tersebut terbagi atas 2 segmen. Segmen pertama menampilkan kawasan waterfront yang terpadu dengan melibatkan segala macam \textit{built environment} di dalamnya. Sedangkan, segmen kedua terlihat seperti desain sebelumnya yakni hanya sebagai wadah pedagang kaki lima yang minim fasilitas.

%Permslhan
Munculnya perbedaan desain pada kawasan tersebut mendorong peneliti untuk mengetahui preferensi (kecenderungan) oleh pengunjung diantara kedua segmen di tepi laut ini dengan terbuka. Menurut \cite{thomas2020}, preferensi suatu pengunjung didasari oleh penghilang stres dan rasa aman. Lebih lanjut, penelitian tersebut menunjukkan bahwa menghilangkan stres adalah faktor signifikan terhadap preferensi responden. Artinya semakin banyak atribut yang mendukung untuk menghilangkan stres di suatu \textit{urban space} akan meningkatkan kecenderungan mereka terhadap \textit{urban space} yang akan ditempatinya.
Dalam \citep{wang2020}, menjelaskan preferensi penduduk kota terhadap \textit{urban space} melalui sejumlah survei kepuasan terhadap kualitas fasilitas yang ada. Responden survei kepuasan itu terdiri dari aspek masyarakat yaitu turis dan penduduk lokal.
Berbeda dengan \cite{imansari2015} yang menghasilkan temuan bahwa preferensi masyarakat didasari oleh kecenderungan sebagai peneduh dan paru-paru kota.
Dari hasil penelitian-penelitian tersebut mengindikasikan bahwa preferensi pengunjung terhadap suatu tempat publik didasari oleh sejumlah alasan yang membuat orang ingin mengunjungi suatu kawasan tepi laut.

\section{Rumusan Masalah}
Tepi laut menjadi sebuah primadona dalam menarik masyarakat perkotaan saat ini. Mereka merupakan aktor utama dalam penyelenggaraan tepi laut yang terpadu. Oleh karena itu, perencanaan sebuah kawasan tepi laut seharusnya mengikuti \textit{preferences} dari khalayak banyak. Revitalisasi yang dilakukan pada kebanyakan waterfront merupakan terobosan bagi perhatian terhadap waterfront, namun terkadang hasil dari Revitalisasi tersebut masih perlu untuk dikaji sesuai dengan aktor(pengguna) yang memanfaatkannya.

Sekelompok pengunjung memanfaatkan sebuah tepi laut dalam berbagai hal seperti wisata kuliner, memancing, \textit{ngopi}, berjalan dan berenang.
Aktivitas yang ada dan belum ada adalah menjadi alasan individu dalam membentuk preferensi mereka. Pengunjung akan mengakomodasi suatu tempat diantara lainnya berdasarkan atribut yang mendukung preferensi mereka. Dengan demikian, penelitian ini merumuskan masalah bagaimana prefensi masyarakat terhadap kedua segmen desain pada kawasan tepi laut.

\section{Tujuan Penelitian}
Penelitian ini bertujuan untuk mengetahui preferensi pengunjung terhadap kawasan tepi laut pantai senggol. Sejumlah atribut mungkin muncul dalam membentuk preferensi pengunjung, seperti aktivitas, kepuasan lingkungan, citra, rasa aman dan penghilang stres.
Preferensi inilah yang menggambarkan kualitas desain yang mana yang paling dominan diminati masyarakat Parepare. Sehingga perencanaan dan konsep desain selanjutnya dapat mengikuti preferensi pengunjung tadi.

\section{Manfaat Penelitian}
Penelitian ini dapat bermanfaat dalam bidang ilmu pengetahuan perencanaan kota di Indonesia. Dalam masa pembangunan infrastruktur sangat dibutuhkan ide atau konsep yang mendukung keberadaan tepi laut yang terpadu dan teruji. Dengan demikian, penelitian ini secara rinci bermanfaat dalam:

\begin{enumerate}
    \item Memberikan masukan desain dengan menggunakan preferensi masyarakat perkotaan dalam mengakomodasi dirinya pada suatu tepi laut.
    \item Memaparkan keadaan pengunjung pantai senggol kota Parepare dalam menggunakan tepi laut yang dapat berguna bagi \textit{stakeholder}, pedagang, atau pemilik kios.
\end{enumerate}

%\subsection{Kerangka Berpikir}




%Preferensi-preferensi yang ada akan  mendorong pengunjung untuk memilih segmen desain sesuai dengan preferensi-preferensi yang terdaftar.

\begin{comment}
\section{Sistematika Penulisan}
Berikut ini adalah sistematika penulisan yang digunakan pada penelitian dimensi kenyamanan pada Waterfront Development:
\begin{itemize}
	\item Bab 1 : Pendahuluan\\
Bab terdiri dari latar belakang permasalahan, perumusan masalah, tujuan penelitian, manfaat penelitian, dan sistematika penulisan.
	\item Bab 2 : Tinjauan Pustaka\\
Bab ini terdiri dari landasan teori yang digunakan untuk memperkuat penemuan masalah, penelitian terdahulu dan kerangka pemikiran.
	\item Bab 3 : Metodologi Penelitian\\
Bab ini terdiri dari penjelasan variabel dan jenis paradigma yang digunakan untuk mencapai penemuan sesuai rumusan masalah, populasi, sampel, dan cara pengumpulan data.
	\item Bab 4 : Hasil dan Pembahasan\\
Bab ini terdiri dari pembahasan mengenai hasil - hasil penelitian yang berupa data-data yang didapatkan, dengan melakukan pengolahan terhadap indikator-indikator kenyamanan. Setelah pengelolahan bahan-bahan tersebut, analisis diperlukan untuk menemukan penemuan penelitian. Analisis diarahkan untuk menjawab rumusan masalah.
	\item Bab V : Kesimpulan\\
Bab terakhir terdiri dari kesimpulan yang didapatkan dari analisis terhadap permasalahan yang terdapat pada penelitian ini, sehingga penemuan bersama saran-saran dari penelusi dapat menghasilkan apa yang diinginkan.


\end{itemize}
\section{Alur Pikir}

\begin{figure}[hp]
\centering
\begin{tikzpicture}[node distance=2cm]
\node (ltr) [startstop] {Latar Belakang};

\node (rum) [startstop, right of=ltr, xshift=2cm] {Perumusan Masalah};

\node (tuj) [startstop, below of=rum, yshift=0.5cm] {Tujuan Penelitian};


\node (pus) [startstop, below of=tuj, yshift=0.5cm] {Studi Pustaka};


\node (kaj) [startstop, below of=pus, text width=3.5cm, xshift= -4cm, yshift=.5cm] {
	\textbf{Kajian Teori}\\ - Fitur binaan\\ - Aktivitas Luar
};


\node (kaj2) [startstop, below of=pus, text width=3.5cm, xshift= 4cm, yshift=.5cm] {
	\textbf{Gambaran Objek}\\ Fitur Binaan dan Aktivitas Luar Jl. Pinggir Laut
};


\node (hip) [startstop, below of=pus, yshift=-.5cm] {Hipotesa};


\node (met) [startstop, below of=hip, yshift=-.75cm, text width=7cm] {
	\textbf{Metode Peneltian}\\ Menggunakan Metode penelitian Kuantitatif Rasionalistik

	\textbf{Variabel}\\
	- Bebas : Fitur Binaan\\
	- Terikat : Aktivitas Luar\\

	\textbf{Sumber data}: Observasi dan Kuesioner
};

\node (ana) [startstop, below of=met, text width=8cm, yshift=-2cm] {
		\textbf{Analisis Data Statistik}\\ Penelitian ini menggunakan metode statika berupa uji regresi guna mengetahui pengaruh variabel fitur binaan terhadap variabel aktivitas luar.
};

\node (tem) [startstop, below of=ana, yshift=-.25cm] {Temuan Penelitian};

\node (kes) [startstop, below of=tem, yshift=.6cm] {Kesimpulan dan Rekomendasi};

\draw [arrow] (ltr) -- (rum);
\draw [arrow] (rum) -- (tuj);
\draw [arrow] (tuj) -- (pus);

\draw [arrow] (pus) -| (kaj);
\draw [arrow] (pus) -| (kaj2);

\draw [doublearrow] (kaj) -- (kaj2);

\draw [arrow] (kaj) |- (met);
\draw [dotted] (kaj) |- (hip);

\draw [arrow] (kaj2) |- (met);
\draw [dotted] (kaj2) |- (hip);

\draw [arrow] (met) -- (ana);
\draw [arrow] (ana) -- (tem);

\draw [arrow] (tem) -- (kes);

\end{tikzpicture}
\caption{Alur Pikir}
\end{figure}

\newpage
\end{comment}
%\onlyinsubfile{\biblio}
\end{document}
