\documentclass[../projects/thesis.tex]{subfiles}

\begin{document}

\chapter{Pendahuluan}\label{chap:pendahuluan}

\section{Latar Belakang}
%developmnet waterfrontis crucial
Tepi laut menjadi sebuah ruang dari perkotaan yang harus terus berkembang \cite{shamsuddin2013}. Kawasan ini memiliki karakteristik dan perhatian khusus mengingat pentingnya air sebagai sumber kehidupan. Untuk mencapai tujuan tersebut, pengembangan tepi laut adalah sangat penting. Menurut \cite{hussein2014}, pengembangan tepi laut yang baik adalah yang mempertimbangkan keberagaman, interaksi komunitas, kenyamanan dan keamanan, lingkungan dan keberlanjutan. Pengembangan berkelanjutan\textit{(sustainable development)} kota pada waterfront mendorong kondisi yang lebih baik untuk masyarakat perkotaan \citep{brebbia2016sustainable}.
Berdasarkan \cite{imperatives1987}, \textit{Sustainability development} adalah pengembangan yang memenuhi kebutuhan saat ini tanpa mempertaruhkan kemampuan dari generasi akan datang untuk memenuhi kebutuhannya.

%pengembngn berdsrkn keberlanjutan adlh kbutuhn
Sebagai negara dengan garis pantai terpanjang di dunia \citep{hindersah2015}, Indonesia memiliki jam terbang yang panjang dalam menghadapi masalah yang rumit dari tepi laut. Pembahasan tentang pengembangan berkelanjutan tepi laut telah ramai diperbincangkan di Indonesia seperti contohnya proyek reklamasi di Makassar dan Manado \citep{andi2017,tungka2012, fhuh2017}, pengembangan ulang tepi laut tahun 1995 sepanjang 32 km di Jakarta \citep{pramesti2017} dan Desain lanskap tepi laut di Sungai Cikapundung \citep{ainy2016}. Menurut \cite{breen1994}, tekanan pada ruang kota dan infrastruktur, kebutuhan atas kualitas lingkungan, dan ketersediaan ruang tepi laut yang terbengkalai menjadi alasan pengembangan ulang kawasan tepi laut sebagai solusi yang pas. Pengembangan ulang tersebut telah di atur sedemikian rupa agar menjadi bagian dari langkah perkotaan yang keberlanjutan \citep{pramesti2017}.

%kondisi parepare
Kota Parepare merupakan kota yang terletak di Provinsi Sulawesi Selatan. Peningkatan jumlah penduduk di Parepare berkisar 2\%, pada tahun 2019 Parepare memiliki penduduk sebanyak 145.178 orang \citep{bpskotaparepare2020}. Dengan mayoritas usia penduduk  merupakan mereka yang berusia awal(0-40). Peningkatan jumlah penduduk tersebut mungkin saja disebabkan oleh potensi Parepare yang menjanjikan untuk kehidupan masyarakat.
Parepare memiliki garis pantai sepanjang 11.8km, lebih kecil dari panjang garis pantai daerah-daerah sekitar seperti Kabupaten Jeneponto (114km) \citep{wardasusaniati2011}, Kabupaten Pangkep(58.87) dan Kabupaten Pinrang(98.51km) \citep{goni2018}. Meskipun demikian, Kota Parepare adalah kota adminsitratif dari 3 kota di Sulawesi Selatan \citep{junaid2016} sebagai faktor pendorong kemajuan kota ini.
Terdapat sejumlah area yang berada di garis pantai tersebut misalnya Tepi Sungai Tonrangeng, Taman Mattirotasi, Pantai Bibir dan Tepi Laut Senggol.
Dengan sejumlah tempat rekreasi tersebut, Parepare mencanangkan konsep kota wisata dengan ikon Patung Bapak BJ Habibie, Presiden Ketiga Republik Indonesia.

%karakteristik di tepi laut senggol
Saat ini, Kota Parepare sedang melakukan sejumlah kemajuan di bidang pariwisata. Salah satunya adalah revitalisasi tepi laut senggol.
Pengembangan tepi laut ini bertujuan agar mampu mendorong jumlah pengunjung pada tempat wisata tersebut, sebagaimana \cite{hoyle2001} menjelaskan keberhasilan suatu tepi laut ditandai dengan pengembangannya membawa masyarakat dan pengunjung untuk datang ke pesisir.
Tepi laut senggol telah lama menjadi daya tarik populer bagi masyarakat setempat. Tepi laut senggol memiliki pemandangan yang sangat indah. Pemandangan sebuah teluk menjadi ciri khas daerah ini. Selain itu, pengunjung juga tertarik untuk berwisata kuliner yang ditemani dengan kombinasi pemandangan elemen daratan dan air. Penyedia utama yang mendukung daya tarik tersebut adalah pedagang kaki lima yang bertahan dari dulu hingga saat ini. Setelah berwisata kuliner, banyak yang menghabiskan sisa waktunya untuk berenang di tepi laut. Menurut \cite{davidowich1998}, bagian yang terpenting dalam pengembangan tepi laut adalah kemampuan pengunjung untuk berinteraksi dengan air. Selain berenang, aktivitas rekreasi seperti memancing dan mencari kepiting membutuhkan akses ke air \citep{gordon1996}. Penggunaan beragam dapat berkontribusi terhadap kesuksessan strategi berkelanjutan \citep{eldeeb2015}.

Tepi laut senggol terbentang dari pelabuhan nusantara hingga pasar Senggol sepanjang 330 meter yang menciptakan sejumlah ruang. Ruang menjadi tempat yang dapat mengakomodasi masyarakat dalam meningkatkan kualitas hidup mereka. Beragam ruang ini memunculkan preferensi ruang masyarakat di kawasan tepi laut. Menurut \cite{devysandra2012}, preferensi adalah kecenderungan untuk memilih sesuatu yang lebih disukai daripada yang lain.

Memunculkan preferensi ruang

Ruang yang ada menjadi

%Keberhasilannya menarik pengunjung mendorong kemajuan bagi objek­objek bangunan sekitar diantara lain: Pasar Senggol(penjualan barang cakar), Pelabuhan, Hanstom(pusat bisnis) dan Bangunan serbaguna lainnya.


Selain sebagai respon terhadap munculnya dua area yang dicenderungi masyarakat, hal ini juga mendukung konsep keberlanjutan yaitu partisipasi masyarakat.
Partisipasi masyarakat juga merupakan elemen penting dalam pengembangan tepi laut berkelanjutan \citep{eldeeb2015,giovinazzi2009}. Kurangnya partisipasi masyarakat dapat membuat taman kota gagal \citep{devysandra2012}. Partisipasi masyarakat bisa dalam berbagai bentuk, satu diantaranya adalah membentuk persetujuan masyarakat \textit{(consensus)} terhadap visi masa depan tepi laut \citep{nysds2009}.


Mengetahui preferensi masyarakat terhadap ruang di tepi laut dapat membantu untuk membentuk peresetujuan itu. Menurut \cite{devysandra2012}, preferensi adalah kecenderungan untuk memilih sesuatu yang lebih disukai daripada yang lain. Sehingga visi masa depan tepi laut dapat jelas dan sesuai keinginan masyarakat.


\section{Rumusan Masalah}
Kawasan tepi laut merupakan kawasan yang sangat rentan dan bernilai tinggi \citep{mullin2000}. Sebagai area yang merupakan bagian yang tidak terpisahkan dari sebuah kota \citep{hussein2014}. Pengembangan tepi laut yang berhasil menarik masyarakat untuk datang ke pesisir. Keberhasilan suatu tepi laut menjadi tanda sebuah kota yang berhasil.
Pengembangan ulang dapat mendongkrak atau menurunkan kualitas suatu tepi laut. Abad 21 ini, Parepare menitikberatkan pembangunan kota dalam aspek kepariwisataan \citep{junaid2016,faniapriani2018,muh.sainals2020} . Lokasi kota Parepare sangat strategis dimana meng­ hubungkan sejumlah kota wisata lainnya di Sulawesi Selatan \citep{junaid2016}, seperti Toraja, Bulukumba, Makassar, dan Palopo. Demikian menjadi pendukung kota Parepare sebagai kota Pariwisata. Potensi ini menjadi alasan perhatian penuh terhadap kawasan tepi laut di pesisir kota Parepare.
Pada tahun 2011, kota Parepare memulai perencanaan penataan pasar senggol hingga kawasan tepi laut senggol. Pengembagan tersebut mendirikan 2 area yang berbeda. Area pertama membangun elemen buatan secara total, sedangkan area kedua memiliki beragaman penggunaan dengan sedikit renovasi. Walaupun area pertama memiliki penataan yang lebih baik, area ini memiliki lebih sedikit pengunjung daripada area kedua. Berlawanan dengan teori \citep{campagnaro2020}, bahwa elemen buatan seperti jalan setapak, kursi, kran air minum berperan penting dalam pemilihan ruang hijau. Berdasarkan permasalahan itu, penelitian ini menyelidiki preferensi masyarakat terhadap ruang di tepi laut senggol. Maka penelitian ini menjawab sejumlah pertanyaan penelitian sebagai berikut:

\begin{enumerate}
    \item Apa fitur­fitur ruang tepi laut yang dicenderungi masyarakat? Apakah kecenderungan ini konsisten diantara kedua area?
    \item Apakah elemen buatan dan beragam penggunaan adalah faktor penting untuk kecenderungan masyarakat terhadap ruang? Apakah kepentingannya bervariasi diantara kedua area?
\end{enumerate}


\section{Tujuan Penelitian}
Penelitian ini bertujuan untuk menjelaskan preferensi masyarakat terhadap ruang tepi laut. Preferensi masyarakat terhadap ruang juga akan dijelaskan dalam konteks pemilihan area di tepi laut senggol Parepare. Partisipasi masyarakat dapat membantu pengembangan dalam mengatasi masalah saat ini dan menjawab tantangan di masa depan. Penulis berharap preferensi masyarakat terhadap ruang tepi laut mendukung pengembangan berkelanjuta di Kota Parepare.


\section{Manfaat Penelitian}

Penelitian ini dapat bermanfaat dalam bidang ilmu pengetahuan peren­ canaan perkotaan di Indonesia. Mengetahui preferensi masyarakat terha­ dap ruang menjadi alat untuk mengikutsertakan masyarakat dalam pengem­ bangan tepi laut. Dalam masa pembangunan infrastruktur Indonesia sangat dibutuhkan pengetahuan yang mendukung kesuksessan tepi laut berkelan­ jutan. Penelitian ini secara detail bermanfaat dalam:

\begin{enumerate}
\item Memberikan masukan desain secara keseluruhan berdasarkan prefe­ rensi masyarakat.
\item Mendukung penelitian selanjutanya dalam ranah preferensi ruang tepi laut.
\item Memberikan panduan terhadap pengembangan tepi laut dimanapun dalam melibatkan masyarakat menggunakan informasi preferensinya.
\end{enumerate}


\begin{comment}
\section{Sistematika Penulisan}
Berikut ini adalah sistematika penulisan yang digunakan pada penelitian dimensi kenyamanan pada Waterfront Development:
\begin{itemize}
	\item Bab 1 : Pendahuluan\\
Bab terdiri dari latar belakang permasalahan, perumusan masalah, tujuan penelitian, manfaat penelitian, dan sistematika penulisan.
	\item Bab 2 : Tinjauan Pustaka\\
Bab ini terdiri dari landasan teori yang digunakan untuk memperkuat penemuan masalah, penelitian terdahulu dan kerangka pemikiran.
	\item Bab 3 : Metodologi Penelitian\\
Bab ini terdiri dari penjelasan variabel dan jenis paradigma yang digunakan untuk mencapai penemuan sesuai rumusan masalah, populasi, sampel, dan cara pengumpulan data.
	\item Bab 4 : Hasil dan Pembahasan\\
Bab ini terdiri dari pembahasan mengenai hasil - hasil penelitian yang berupa data-data yang didapatkan, dengan melakukan pengolahan terhadap indikator-indikator kenyamanan. Setelah pengelolahan bahan-bahan tersebut, analisis diperlukan untuk menemukan penemuan penelitian. Analisis diarahkan untuk menjawab rumusan masalah.
	\item Bab V : Kesimpulan\\
Bab terakhir terdiri dari kesimpulan yang didapatkan dari analisis terhadap permasalahan yang terdapat pada penelitian ini, sehingga penemuan bersama saran-saran dari penelusi dapat menghasilkan apa yang diinginkan.


\end{itemize}
\section{Alur Pikir}

\begin{figure}[hp]
\centering
\begin{tikzpicture}[node distance=2cm]
\node (ltr) [startstop] {Latar Belakang};

\node (rum) [startstop, right of=ltr, xshift=2cm] {Perumusan Masalah};

\node (tuj) [startstop, below of=rum, yshift=0.5cm] {Tujuan Penelitian};


\node (pus) [startstop, below of=tuj, yshift=0.5cm] {Studi Pustaka};


\node (kaj) [startstop, below of=pus, text width=3.5cm, xshift= -4cm, yshift=.5cm] {
	\textbf{Kajian Teori}\\ - Fitur binaan\\ - Aktivitas Luar
};


\node (kaj2) [startstop, below of=pus, text width=3.5cm, xshift= 4cm, yshift=.5cm] {
	\textbf{Gambaran Objek}\\ Fitur Binaan dan Aktivitas Luar Jl. Pinggir Laut
};


\node (hip) [startstop, below of=pus, yshift=-.5cm] {Hipotesa};


\node (met) [startstop, below of=hip, yshift=-.75cm, text width=7cm] {
	\textbf{Metode Peneltian}\\ Menggunakan Metode penelitian Kuantitatif Rasionalistik

	\textbf{Variabel}\\
	- Bebas : Fitur Binaan\\
	- Terikat : Aktivitas Luar\\

	\textbf{Sumber data}: Observasi dan Kuesioner
};

\node (ana) [startstop, below of=met, text width=8cm, yshift=-2cm] {
		\textbf{Analisis Data Statistik}\\ Penelitian ini menggunakan metode statika berupa uji regresi guna mengetahui pengaruh variabel fitur binaan terhadap variabel aktivitas luar.
};

\node (tem) [startstop, below of=ana, yshift=-.25cm] {Temuan Penelitian};

\node (kes) [startstop, below of=tem, yshift=.6cm] {Kesimpulan dan Rekomendasi};

\draw [arrow] (ltr) -- (rum);
\draw [arrow] (rum) -- (tuj);
\draw [arrow] (tuj) -- (pus);

\draw [arrow] (pus) -| (kaj);
\draw [arrow] (pus) -| (kaj2);

\draw [doublearrow] (kaj) -- (kaj2);

\draw [arrow] (kaj) |- (met);
\draw [dotted] (kaj) |- (hip);

\draw [arrow] (kaj2) |- (met);
\draw [dotted] (kaj2) |- (hip);

\draw [arrow] (met) -- (ana);
\draw [arrow] (ana) -- (tem);

\draw [arrow] (tem) -- (kes);

\end{tikzpicture}
\caption{Alur Pikir}
\end{figure}

\newpage
\end{comment}
%\onlyinsubfile{\biblio}
\end{document}
