\documentclass[../projects/thesis.tex]{subfiles}

\begin{document}

\chapter{Pendahuluan}\label{chap:pendahuluan}

\section{Latar Belakang}
%Sustainability developmnet waterfront
Tepi laut menjadi sebuah ruang dari perkotaan yang harus terus berkembang \cite{shamsuddin2013}. Kawasan ini memiliki karakteristik dan perhatian khusus mengingat pentingnya air sebagai sumber kehidupan. Untuk mencapai tujuan tersebut, pengembangan tepi laut adalah sangat penting. Menurut \cite{hussein2014}, pengembangan tepi laut yang baik adalah yang mempertimbangkan keberagaman, interaksi komunitas, kenyamanan dan keamanan, lingkungan dan keberlanjutan. Pengembangan berkelanjutan\textit{(sustainable development)} kota pada waterfront mendorong kondisi yang lebih baik untuk masyarakat perkotaan \citep{brebbia2016sustainable}.
Berdasarkan \cite{imperatives1987}, \textit{Sustainability development} adalah pengembangan yang memenuhi kebutuhan saat ini tanpa mempertaruhkan kemampuan dari generasi akan datang untuk memenuhi kebutuhannya.

%ltar belkng sustain indonesia
Sebagai negara dengan garis pantai terpanjang di dunia \citep{hindersah2015}, Indonesia memiliki jam terbang yang panjang dalam menghadapi masalah yang rumit dari tepi laut. Pembahasan tentang pengembangan berkelanjutan tepi laut telah ramai diperbincangkan di Indonesia seperti contohnya proyek reklamasi di Makassar dan Manado \citep{andi2017,tungka2012, fhuh2017}, pengembangan ulang tepi laut tahun 1995 sepanjang 32 km di Jakarta \citep{pramesti2017} dan Desain lanskap tepi laut di Sungai Cikapundung \citep{ainy2016}. Menurut \cite{breen1993}, tekanan pada ruang kota dan infrastruktur, kebutuhan atas kualitas lingkungan, dan ketersediaan ruang tepi laut yang terbengkalai menjadi alasan pengembangan ulang kawasan tepi laut sebagai solusi yang pas. Pengembangan ulang tersebut telah di atur sedemikian rupa agar menjadi bagian dari langkah perkotaan yang keberlanjutan \citep{pramesti2017}.

%kondisi parepare
Kota Parepare merupakan kota yang terletak di Provinsi Sulawesi Selatan. Peningkatan jumlah penduduk di Parepare berkisar 2\%, pada tahun 2019 Parepare memiliki penduduk sebanyak 145.178 orang \citep{bpskotaparepare2020}. Dengan mayoritas usia penduduk  merupakan mereka yang berusia awal(0-40). Peningkatan jumlah penduduk tersebut mungkin saja disebabkan oleh potensi Parepare yang sangat menjanjikan di berbagai sektor.
Parepare memiliki garis pantai sepanjang 11.8km, lebih kecil dari panjang garis pantai daerah-daerah sekitar seperti Kabupaten Jeneponto (114km) \citep{wardasusaniati2011}, Kabupaten Pangkep(58.87) dan Kabupaten Pinrang(98.51km) \citep{goni2018}. Meskipun demikian, Kota Parepare adalah kota adminsitratif dari 3 kota di Sulawesi Selatan \citep{junaid2016} sebagai faktor pendorong kemajuan kota ini.
Terdapat sejumlah area yang berada di garis pantai tersebut misalnya Tepi Sungai Tonrangeng, Taman Mattirotasi, Pantai Bibir dan Tepi Laut Senggol.
Dengan sejumlah tempat rekreasi tersebut, Parepare mencanangkan konsep kota wisata dengan ikon Patung Bapak BJ Habibie, Presiden Ketiga Republik Indonesia.

%city scale to area scale
%redevelopmnet waterfront
Saat ini, Kota Parepare sedang melakukan sejumlah kemajuan di bidang pariwisata. Salah satunya adalah revitalisasi ruang publik di tepi laut senggol.
Pengembangan tepi laut ini bertujuan agar mampu mendorong jumlah pengunjung pada tempat wisata tersebut, sebagaimana \cite{hoyle2001} menjelaskan keberhasilan suatu tepi laut ditandai dengan pengembangannya membawa masyarakat dan pengunjung untuk datang ke pesisir.
Tepi laut senggol telah lama menjadi daya tarik utama penduduk maupun pengunjung. Keberhasilannya menarik pengunjung mendorong kemajuan bagi objek-objek bangunan sekitar diantara lain: Pasar Senggol(perdagangan barang cakar), Pelabuhan, Hanstom(pusat bisnis) dan Bangunan serbaguna lainnya.
Meskipun tepi laut senggol sudah dapat membawa dirinya hingga saat ini, tantangan terhadap waterfront akan datang silih berganti. Bahkan dalam skala besar, pada abad 21, kota dihadapkan dengan berbagai tantangan yang tidak diketahui kuantitasnya yang sangat perlu dijawab \citep{niemann2016}.


Pengembawangan w
Namun aspek terlupakan adalah keberlanjutan.
Untuk waterfront harus ada keberlanjutan.

Ruang publik menjadi kunci elemen dalam pengembangan keberlanjutan.

Kawasan tepi laut senggol diyakini dapat memenuhi kebutuhan terhadap pengembangan berkelanjutan.
%%
%waterfront bagus membutuhkn pengembgn berkelanjutan
%bagmna waterfrong bagus untuk berkelanjutan

%Menurut \cite{hussein2014}, menghubungkan objek yang lama dan baru serta mengikutsertakan masyarakat dapat berkontribusi dalam kesuksessan pengembangan waterfront. Sehingga menjaga masa lalu kawasan tepi laut senggol ini penting untuk melengkapi pengembangan berkelanjutan \citep{giovinazzi2008}.


Dengan begitu, berbagai kalangan masyarakat dapat menikmati beragam penggunaan yang ada pada waterfront.
Hal demikian dapat mendukung penyelenggaran konsep berkelanjutan.
Kawasan tepi laut senggol merupakan daya tarik cukup lama yang ada di Kota Parepare.
Tempat ini berdampingan dengan sejumlah objek wisata kota yang meninggalkan banyak warisan. Waterfront yang sukses adalah
Untuk menciptkan waterfront yang dapat menarik pengunjung, waterfront harus mempertimbangkan
Agar menciptakan revitalisasi yang baik, waterfron
%preserve old and add new(principle)
Dengan konsep keberlanjutan ini, kawasan tepi laut diharapkan mampu menjembatani masyarakat dengan kecederungan terhadap objek yang telah ada dan pengembangan baru.
%public access(principle)
Lokasi Parepare ada
strategis
Parepare sebagai kota transit di wilayah Sulawesi Selatan menarik pengunjung dari latar belakang yang berbeda ke tempat wisata.

%--------------------------------------------------------------------------------------
%Kawasan tepi laut senggol menjadi
%Namun, pengembangan berkelanjutan
%redevelopmnt wtrfron hrus menjaga warisan perkotaan dan menerapkan pengembangan baru.
%
%Menurut \cite{giovinazzi2008}, melalui beragam penggunaaan pada kawasan tepi laut, macam-macam pengunjung akan tertarik untuk datang.
%
%Pengembangan waterfront menekankan agar proyek revitalisasi dapat mengakomodasi aspek-aspek seperti budaya, sosial, komersial dan beragam gaya hidup secara menyeluruh \citep{niemann2016}.
%
%Alhasil, ruang publik merupakan sesuatu yang sangat penting dalam pengembangan berkelanjutan.



%Revitalisasi ini berpengaruh terhadap keadaa
%
%keberlangsungan lingkungan dan keadaan socio-economic pada tempat-tempat tersebut.
%
%Beberapa diantaranya berada di kawasan pesisir laut.
%
%redevelopmen menuju kwasan wisata (mengatasi socio economy)

% Redevelopment pinggir laut parepare
Pada tahun 2011, kota Parepare memulai perencanaan penataan pasar senggol serta jalan kawasan pinggir laut. Hal ini untuk merespon tujuan kota sebagai kota wisata yang berujung dalam peningkatan ekonomi dan nilai sebuah kota. Revitalisasi tersebut memunculkan beragam desain yang berhubungan dengan  preferensi pengunjung terhadap kawasan. Secara garis besar, pembagian macam kawasan tersebut terbagi atas 2 segmen. Segmen pertama menampilkan kawasan waterfront yang terpadu dengan melibatkan segala macam \textit{built environment} di dalamnya. Sedangkan, segmen kedua terlihat seperti desain sebelumnya yakni hanya sebagai wadah pedagang kaki lima yang minim fasilitas.

%Permslhan
Munculnya perbedaan desain pada kawasan tersebut mendorong peneliti untuk mengetahui preferensi (kecenderungan) oleh pengunjung diantara kedua segmen di tepi laut ini dengan terbuka. Menurut \cite{thomas2020}, preferensi suatu pengunjung didasari oleh penghilang stres dan rasa aman. Lebih lanjut, penelitian tersebut menunjukkan bahwa menghilangkan stres adalah faktor signifikan terhadap preferensi responden. Artinya semakin banyak atribut yang mendukung untuk menghilangkan stres di suatu \textit{urban space} akan meningkatkan kecenderungan mereka terhadap \textit{urban space} yang akan ditempatinya.
Dalam \citep{wang2020}, menjelaskan preferensi penduduk kota terhadap \textit{urban space} melalui sejumlah survei kepuasan terhadap kualitas fasilitas yang ada. Responden survei kepuasan itu terdiri dari aspek masyarakat yaitu turis dan penduduk lokal.
Berbeda dengan \cite{imansari2015} yang menghasilkan temuan bahwa preferensi masyarakat didasari oleh kecenderungan sebagai peneduh dan paru-paru kota.
Dari hasil penelitian-penelitian tersebut mengindikasikan bahwa preferensi pengunjung terhadap suatu tempat publik didasari oleh sejumlah alasan yang membuat orang ingin mengunjungi suatu kawasan tepi laut.



\begin{comment}
%THE IMPORTANCE RIVERSIDE IN GENERAL
Menurut \cite{shamsuddin2013}, riverside merupakan ruang perkotaan yang harus terus berkembang. Kawasan inilah yang diberkahi dengan karakteristik dan perhatian khusus mengingat pentingnya air sebagai sumber kehidupan kota.
Pada area laut perkotaan, lomba untuk ruang waterfront, kebutuhan publik untuk mengakses pesisir laut dan mempertahankan biodiversity tepi laut sebagai sumber alami menjadi isu terhangat dalam kebijakan perkotaan \citep{breen1994waterfronts}.
\end{comment}
%KEGAGALAN WATERFRONT IN GENERAL
Tidak hanya memberikan sebuah ruang pada suatu lahan di perkotaan, tetapi interaksi antara ruang dan pengguna adalah sangat penting. Pengembangan tepi laut seharusnya yang menciptakan ruang publik yang menyenangkan. Banyak kota yang telah gagal dalam memberikan kesan pada sebuah Waterfront, pandangan orang menjadi sosok yang kotor, padat, bahkan bahaya di sebuah sudut kota. Dalam pandangan \cite{goodwin1999} tepi laut biasa dipersepsikan sebagai sesuatu yang samar dan tercampur dengan area yang diabaikan, pusat komersial dan permukiman.

%PENURUNAN KUALITAS
Seiring perkembangan kota, waterfront seringkali mengalami kegoyahan dalam perkembangannya. Beberapa hal yang menyebabkan penurunan kualitas  waterfront adalah adanya peningkatan mobilitas yang melewati area tersebut\citep{richarda.lehmann1966}. \cite{ulam2009} menjelaskan kawasan waterfront pada paruh kedua abad ke-20 Waterfront harlem mengalami penurunan kualitas ditandai tempat hanya menjadi tempat memancing, penggunaan narkoba dan prostitusi. Bahkan menerima persepsi yang buruk oleh penduduk, tidak dapat di akses, \textit{privatisasi} bank, kontaminasi air dan pembukaan jalur tol paralel dengan pesisir \citep{shamsuddin2013}. Menurut \cite{benages2015revisiting} tepi laut mengalami seperti penurunan yang signifikan disebabkan oleh berbagai faktor seperti tekanan \textit{real estate} dan perencanaan yang kurang. Kejadian ini membuat pemerintah kesusahan dalam mengatur pengembangan tepi laut \citep{gripaios1999ports}.


%REDEVELOPMENT
Meski pun demikian, pengembangan waterfront kian meningkat dari abad 19. Misalnya tepi laut kota Witconsin yang mengalami kegagalan dalam memberikan rasa ruang dan perbedaan kota itu akhirnya berubah. Dengan konsep `Waterfront Renewal' melibatkan 50 kota di Amerika \citep{richarda.lehmann1966}. Ditambah lagi dengan `waterfront redevelopment' pada pelabuhan Thessaloniki di Yunani. Serta di Liverpool, inggris sekitar tahun 1940 \citep{couch2003city}. Adanya peningkatan ini memberikan harapan dan potensi untuk pengembangan tepi laut di tempat lainnya.
%WATERFRONT RESTORE CITY
Pesisir laut saat ini menjadi fokus pengembangan dalam sebuah kota. Dalam penelitian pelabuhan dan tepi laut industri di Thessaloniki, Yunani. Meski pun dalam keadaan ekonomi terpuruk, mereka masih menyediakan area yang cukup luas untuk bangunan mewah baru yang berdiri di pinggir laut. Pembangunan tersebut kebanyakan berada di pusat kota dan menjadi simbol suatu ekonomi 'sukses' \citep{vayona2011}.

% Redevelopment pinggir laut parepare
Pada tahun 2011, kota Parepare memulai perencanaan penataan pasar senggol serta jalan kawasan pinggir laut. Hal ini untuk merespon tujuan kota sebagai kota wisata yang berujung dalam peningkatan ekonomi dan nilai sebuah kota. Revitalisasi tersebut memunculkan beragam desain yang berhubungan dengan  preferensi pengunjung terhadap kawasan. Secara garis besar, pembagian macam kawasan tersebut terbagi atas 2 segmen. Segmen pertama menampilkan kawasan waterfront yang terpadu dengan melibatkan segala macam \textit{built environment} di dalamnya. Sedangkan, segmen kedua terlihat seperti desain sebelumnya yakni hanya sebagai wadah pedagang kaki lima yang minim fasilitas.

%Permslhan
Munculnya perbedaan desain pada kawasan tersebut mendorong peneliti untuk mengetahui preferensi (kecenderungan) oleh pengunjung diantara kedua segmen di tepi laut ini dengan terbuka. Menurut \cite{thomas2020}, preferensi suatu pengunjung didasari oleh penghilang stres dan rasa aman. Lebih lanjut, penelitian tersebut menunjukkan bahwa menghilangkan stres adalah faktor signifikan terhadap preferensi responden. Artinya semakin banyak atribut yang mendukung untuk menghilangkan stres di suatu \textit{urban space} akan meningkatkan kecenderungan mereka terhadap \textit{urban space} yang akan ditempatinya.
Dalam \citep{wang2020}, menjelaskan preferensi penduduk kota terhadap \textit{urban space} melalui sejumlah survei kepuasan terhadap kualitas fasilitas yang ada. Responden survei kepuasan itu terdiri dari aspek masyarakat yaitu turis dan penduduk lokal.
Berbeda dengan \cite{imansari2015} yang menghasilkan temuan bahwa preferensi masyarakat didasari oleh kecenderungan sebagai peneduh dan paru-paru kota.
Dari hasil penelitian-penelitian tersebut mengindikasikan bahwa preferensi pengunjung terhadap suatu tempat publik didasari oleh sejumlah alasan yang membuat orang ingin mengunjungi suatu kawasan tepi laut.

\section{Rumusan Masalah}
Tepi laut menjadi sebuah primadona dalam menarik masyarakat perkotaan saat ini. Mereka merupakan aktor utama dalam penyelenggaraan tepi laut yang terpadu. Oleh karena itu, perencanaan sebuah kawasan tepi laut seharusnya mengikuti \textit{preferences} dari khalayak banyak. Revitalisasi yang dilakukan pada kebanyakan waterfront merupakan terobosan bagi perhatian terhadap waterfront, namun terkadang hasil dari Revitalisasi tersebut masih perlu untuk dikaji sesuai dengan aktor(pengguna) yang memanfaatkannya.

Sekelompok pengunjung memanfaatkan sebuah tepi laut dalam berbagai hal seperti wisata kuliner, memancing, \textit{ngopi}, berjalan dan berenang.
Aktivitas yang ada dan belum ada adalah menjadi alasan individu dalam membentuk preferensi mereka. Pengunjung akan mengakomodasi suatu tempat diantara lainnya berdasarkan atribut yang mendukung preferensi mereka. Dengan demikian, penelitian ini merumuskan masalah bagaimana prefensi masyarakat terhadap kedua segmen desain pada kawasan tepi laut.

\section{Tujuan Penelitian}
Penelitian ini bertujuan untuk mengetahui preferensi pengunjung terhadap kawasan tepi laut pantai senggol. Sejumlah atribut mungkin muncul dalam membentuk preferensi pengunjung, seperti aktivitas, kepuasan lingkungan, citra, rasa aman dan penghilang stres.
Preferensi inilah yang menggambarkan kualitas desain yang mana yang paling dominan diminati masyarakat Parepare. Sehingga perencanaan dan konsep desain selanjutnya dapat mengikuti preferensi pengunjung tadi.

\section{Manfaat Penelitian}
Penelitian ini dapat bermanfaat dalam bidang ilmu pengetahuan perencanaan kota di Indonesia. Dalam masa pembangunan infrastruktur sangat dibutuhkan ide atau konsep yang mendukung keberadaan tepi laut yang terpadu dan teruji. Dengan demikian, penelitian ini secara rinci bermanfaat dalam:

\begin{enumerate}
    \item Memberikan masukan desain dengan menggunakan preferensi masyarakat perkotaan dalam mengakomodasi dirinya pada suatu tepi laut.
    \item Memaparkan keadaan pengunjung pantai senggol kota Parepare dalam menggunakan tepi laut yang dapat berguna bagi \textit{stakeholder}, pedagang, atau pemilik kios.
\end{enumerate}

%\subsection{Kerangka Berpikir}




%Preferensi-preferensi yang ada akan  mendorong pengunjung untuk memilih segmen desain sesuai dengan preferensi-preferensi yang terdaftar.

\begin{comment}
\section{Sistematika Penulisan}
Berikut ini adalah sistematika penulisan yang digunakan pada penelitian dimensi kenyamanan pada Waterfront Development:
\begin{itemize}
	\item Bab 1 : Pendahuluan\\
Bab terdiri dari latar belakang permasalahan, perumusan masalah, tujuan penelitian, manfaat penelitian, dan sistematika penulisan.
	\item Bab 2 : Tinjauan Pustaka\\
Bab ini terdiri dari landasan teori yang digunakan untuk memperkuat penemuan masalah, penelitian terdahulu dan kerangka pemikiran.
	\item Bab 3 : Metodologi Penelitian\\
Bab ini terdiri dari penjelasan variabel dan jenis paradigma yang digunakan untuk mencapai penemuan sesuai rumusan masalah, populasi, sampel, dan cara pengumpulan data.
	\item Bab 4 : Hasil dan Pembahasan\\
Bab ini terdiri dari pembahasan mengenai hasil - hasil penelitian yang berupa data-data yang didapatkan, dengan melakukan pengolahan terhadap indikator-indikator kenyamanan. Setelah pengelolahan bahan-bahan tersebut, analisis diperlukan untuk menemukan penemuan penelitian. Analisis diarahkan untuk menjawab rumusan masalah.
	\item Bab V : Kesimpulan\\
Bab terakhir terdiri dari kesimpulan yang didapatkan dari analisis terhadap permasalahan yang terdapat pada penelitian ini, sehingga penemuan bersama saran-saran dari penelusi dapat menghasilkan apa yang diinginkan.


\end{itemize}
\section{Alur Pikir}

\begin{figure}[hp]
\centering
\begin{tikzpicture}[node distance=2cm]
\node (ltr) [startstop] {Latar Belakang};

\node (rum) [startstop, right of=ltr, xshift=2cm] {Perumusan Masalah};

\node (tuj) [startstop, below of=rum, yshift=0.5cm] {Tujuan Penelitian};


\node (pus) [startstop, below of=tuj, yshift=0.5cm] {Studi Pustaka};


\node (kaj) [startstop, below of=pus, text width=3.5cm, xshift= -4cm, yshift=.5cm] {
	\textbf{Kajian Teori}\\ - Fitur binaan\\ - Aktivitas Luar
};


\node (kaj2) [startstop, below of=pus, text width=3.5cm, xshift= 4cm, yshift=.5cm] {
	\textbf{Gambaran Objek}\\ Fitur Binaan dan Aktivitas Luar Jl. Pinggir Laut
};


\node (hip) [startstop, below of=pus, yshift=-.5cm] {Hipotesa};


\node (met) [startstop, below of=hip, yshift=-.75cm, text width=7cm] {
	\textbf{Metode Peneltian}\\ Menggunakan Metode penelitian Kuantitatif Rasionalistik

	\textbf{Variabel}\\
	- Bebas : Fitur Binaan\\
	- Terikat : Aktivitas Luar\\

	\textbf{Sumber data}: Observasi dan Kuesioner
};

\node (ana) [startstop, below of=met, text width=8cm, yshift=-2cm] {
		\textbf{Analisis Data Statistik}\\ Penelitian ini menggunakan metode statika berupa uji regresi guna mengetahui pengaruh variabel fitur binaan terhadap variabel aktivitas luar.
};

\node (tem) [startstop, below of=ana, yshift=-.25cm] {Temuan Penelitian};

\node (kes) [startstop, below of=tem, yshift=.6cm] {Kesimpulan dan Rekomendasi};

\draw [arrow] (ltr) -- (rum);
\draw [arrow] (rum) -- (tuj);
\draw [arrow] (tuj) -- (pus);

\draw [arrow] (pus) -| (kaj);
\draw [arrow] (pus) -| (kaj2);

\draw [doublearrow] (kaj) -- (kaj2);

\draw [arrow] (kaj) |- (met);
\draw [dotted] (kaj) |- (hip);

\draw [arrow] (kaj2) |- (met);
\draw [dotted] (kaj2) |- (hip);

\draw [arrow] (met) -- (ana);
\draw [arrow] (ana) -- (tem);

\draw [arrow] (tem) -- (kes);

\end{tikzpicture}
\caption{Alur Pikir}
\end{figure}

\newpage
\end{comment}
%\onlyinsubfile{\biblio}
\end{document}
